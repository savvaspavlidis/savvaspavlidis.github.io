\documentclass[a4paper,12pt]{article}                      % Ορίζουμε την γεωμετρία του χαρτιού και του διαθέσιμου χώρου
\usepackage[a4paper]{geometry}
\geometry{top=0.8in,bottom=0.8in,left=1.0in,right=1.0in}   % Τα παρακάτω είναι τα βασικά για το XeLATEX 
\usepackage{sectsty}
\usepackage{fontspec}
\usepackage{xunicode}
\usepackage{xltxtra}
% \usepackage{xgreek}
\usepackage{graphicx}                                       % Επιλέγουμε την κύρια γραμματοσειρά.
\setmainfont[Mapping=tex-text]{GFS Didot}                   % Πακέτο που μας δίνει την δυνατότητα ορισμού διαστίχου σε αλλαγή παραγράφου
\usepackage{parskip}
\usepackage{fancyvrb}
\usepackage[numbers]{natbib}
\usepackage{adjustbox}
\usepackage{array}
\usepackage{booktabs}
\usepackage{multirow}
\usepackage[usenames, dvipsnames]{color}
\usepackage{colortbl}
\usepackage{endnotes}
% \usepackage{breqn}                                            % Για την δυνατότητα headers/footers με πολλές δυνατότητες όπως πχ τοποθέτηση εικόνων κλπ
% \usepackage{fancyhdr}                                         % Αυτό το χρησιμοποιούμε για να έχουμε την δυνατότητα ακριβούς
%                                                               % τοποθέτησης πινάκων και εικόνων / γραφημάτων
\usepackage{float}                                          % Βάζουμε ένα όριο δικό μας σχετικά μεγάλο για να μην μας τρελλένει με warnings για underfull boxes
\usepackage{amsmath}
\usepackage{amsfonts}
\usepackage{enumerate}
\usepackage{draftwatermark}
\usepackage{multicol}
\usepackage{bigints}
% \hbadness=3000
% \hfuzz=5pt                                                    % Ορίζουμε τα μεγέθη για τις επικεφαλίδες ανάλογα.
\sectionfont{\fontsize{16.0pt}{19.2pt}\selectfont\raggedright}
\subsectionfont{\fontsize{14.0pt}{16.8pt}\selectfont\raggedright}
\subsubsectionfont{\fontsize{12.0pt}{14.4pt}\selectfont\raggedright}
\usepackage{setspace}
\onehalfspacing
\parindent 0.2in                                            % Αυτό χρειάζεται για τα headers/footers
% \pagestyle{fancy}
\usepackage{pgf,tikz}
\usepackage{tkz-tab}
\usepackage{pgfplots}

\usetikzlibrary{positioning,shapes,shadows,arrows}

\usepackage[makeroom]{cancel}


\def\wl{\par \vspace{\baselineskip}} 						% Ορίζουμε ένα δικό μας macro, για αλλαγή γραμμής


\newcolumntype{R}[2]{%
    >{\adjustbox{angle=#1,lap=\width-(#2)}\bgroup}%
    l%
    <{\egroup}%
}
\newcommand*\rot{\multicolumn{1}{R{90}{1em}}}% no optional argument here, please!
\renewcommand{\arraystretch}{1.3}

\let\oldsqrt\sqrt 							% it defines the new \sqrt in terms of the old one
\def\sqrt{\mathpalette\DHLhksqrt}
\def\DHLhksqrt#1#2{%
\setbox0=\hbox{$#1\oldsqrt{#2\,}$}\dimen0=\ht0
\advance\dimen0-0.2\ht0
\setbox2=\hbox{\vrule height\ht0 depth -\dimen0}%
{\box0\lower0.4pt\box2}}

\newcolumntype{L}[1]{>{\raggedright\let\newline\\\arraybackslash\hspace{0pt}}m{#1}}
\newcolumntype{C}[1]{>{\centering\let\newline\\\arraybackslash\hspace{0pt}}m{#1}}
\newcolumntype{R}[1]{>{\raggedleft\let\newline\\\arraybackslash\hspace{0pt}}m{#1}}


\begin{document}
\SetWatermarkFontSize{90pt}					% \SetWatermarkScale{1.5}
\SetWatermarkLightness{0.9}
\SetWatermarkText{1\textsuperscript{0} ΕΠΑΛ Πολίχνης}

\begin{center}
\LARGE{ΣΗΜΕΙΩΣΕΙΣ ΣΤΑΤΙΣΤΙΚΗΣ}
\end{center}

\section{Ορισμοί}
\begin{multicols}{2}
\textbf{Στατιστική} ειναι το σύνολο αρχών και μεθοδολογίας για:
\begin{enumerate}
\item Σχεδιασμός της διαδικασίας συλλογής δεδομένων
\item Συνοπτική και αποτελεσματική παρουσίαση των δεδομένων
\item Ανάλυση των δεδομένων και εξαγωγή συμπερασμάτων
\end{enumerate}

\textbf{Χρησιμότητα της Στατιστικής}
\begin{enumerate}
\item Κατανόηση του παρελθόντων (παρελθόντων δεδομένων)
\item Πρόβλεψη του μέλλοντος
\end{enumerate}

\textbf{Πληθυσμός} ονομάζεται το σύνολο των οντοτήτων  του οποία τα στοιχεία εξετάζουμε ως προς ένα ή περισσότερα χαρακτηριστικά

\textbf{Μονάδες ή στοιχεία του πληθυσμού} ονομάζονται τα στοιχεία του πληθυσμού

\textbf{Μεταβλητή} ονομάζεται το χαρακτη-ριστικό ως προς το οποίο εξετάζουμε τις μονάδες ή άτομα ενός πληθυσμού

\textbf{Στατιστικά δεδομένα ή παρατηρήσεις} ονομάζονται τα δεδομένα (αποτελέσματα) απο την παρατήρηση του χαρακτηριστικού (μεταβλητής) σε κάθε μονάδα ή άτομο του πληθυσμού. Συνήθως συμβολίζονται ως $ t_1, t_2, t_3,... $

Οι μεταβλητές χωριζονται σε δύο μεγάλες κατηγορίες, \textbf{ποιοτικές} ή \textbf{ποσοτικές}. \textbf{Ποιοτική} ή κατηγορική (απο το κατηγορία) είναι η μεταβλητή που οι τιμές της δεν είναι αριθμοί, ή εαν είναι, δεν μπορούν να εννοηθούν μέτρα θέσης (πχ μέση τιμή). \textbf{Ποσοτική} είναι η μεταβλητή που οι τιμές της είναι αριθμοί, και εννοούνται μέτρα θέσης (πχ μέση τιμή)

Οι ποσοτικές μεταβλητές διακρίνονται σε δυο κατηγορίες, τις \textbf{διακριτές} και τις \textbf{συνεχείς}. \textbf{Συνεχείς} μεταβλητές είναι αυτές που μπορούν να πάρουν οποιαδήποτε τιμή ενός διαστήματος πραγματικών αριθμών. \textbf{Διακριτές} μεταβλητές είναι αυτές οι οποίες "παίρνουν" μεμονωμένες τιμές (δηλαδή το σύνολο των υπαρκτών διαφορετικών τιμών είναι μικρό, πχ μικρότερο απο 10-20 διαφορετικές τιμές)

\textbf{Δειγμα} ονομάζεται ένα υποσύνολο του πληθυσμού το οποίο εξετάζεται ως προς το χαρακτηριστικό που μας ενδιαφέρει.

\textbf{Αντιπροσωπευτικό δείγμα} ονομάζεται το δείγμα ενός πληθυσμού, εαν έχει επιλεγεί με τέτοιο τρόπο ώστε κάθε μονάδα του πληθυσμού να έχει την ίδια πιθανότητα να επιλεγεί.

\textbf{Δειγματοληψία} ονομάζεται η διαδικασία επιλογής ενός δείγματος.

\textbf{Μέγεθος δείγματος} ονομάζεται το πλήθος των παρατηρήσεων του δείγματος. Ειναι πάντα θετικός αριθμός και συμβολίζεται με το γράμμα $ \nu $

\textbf{Στατιστικοί Πίνακες}
Για να είναι σωστά κατασκευασμένος ένας στατιστικός πίνακας πρέπει να περιέχει:
\begin{enumerate}
\item \textbf{Τίτλος}. Γράφεται στο πάνω μέρος του πίνακα και δηλώνει με σαφήνεια και συνοπτικά τι περιέχει ο πίνακας
\item \textbf{Επικεφαλίδα}. Γράφεται σε κάθε γραμμή ή στήλη και δείχνει συνοπτικά την φύση και την μονάδα μέτρησης των δεδομένων
\item \textbf{Κύριο σώμα}. Περιέχει, διαχωρισμένα μέσα σε γραμμές και στήλες τα στατιστικά δεδομένα
\item \textbf{Πηγή}. Γράφεται στο κάτω μέρος του πίνακα και δείχνει την προέλευση των στατιστικών στοιχείων (πχ για επαλήθευση ή περαιτέρω έρευνα).
\end{enumerate}

Ο φυσικός αριθμός που δείχνει πόσες φορές εμφανίζεται η τιμή $ x_i, i=1,2,...,\kappa $ της εξεταζόμενης μεταβλητής, λέγεται \textbf{συχνότητα} και συμβολίζεται $ \nu_i, i=1,2,...,\kappa $

\textbf{Σχετική συχνότητα $ f_i $ } της τιμής $ x_i $ ορίζεται ως $ f_i=\dfrac{\nu_i}{\nu}, i=1,2,...,\kappa $ 

\textbf{Πίνακας Κατανομής Συχνοτήτων} ονομάζεται ο συνοπτικός πίνακας που έχει τα στοιχεία όπως οι διαφορετικές περιπτώσεις της μεταβλητής $ x_i $ καθώς και τις στήλες $ \nu_i $ και $ f_i $.

Οι ομάδες στις οποίες ταξινομούνται οι τιμές ονομάζονται \textbf{κλάσεις} και ειναι της μορφής $ [\alpha,\beta) $. Τα άκρα της κλάσης, δηλαδή τα $\alpha$ και $\beta$ ονομάζονται \textbf{όρια των κλάσεων}.

Το κέντρο κάθε κλάσης λέγεται \textbf{κεντρική τιμή $x_i$ }. Το κέντρο της κλάσης $[\alpha,\beta) $ δίδεται απο τον τύπο $ x_i=\dfrac{\alpha+\beta}{2} $

\textbf{Πλάτος c μιας κλάσης} ονομάζεται η διαφορά του κατώτερου απο το ανώτερο όριο της κλάσης. Δηλαδή για μια κλάση $[\alpha,\beta) $ αυτό ειναι $ c=\beta-\alpha $

\textbf{Ευρος R} του δείγματος ονομάζεται η διαφορά της μικρότερης παρατήρησης απο την μεγαλύτερη παρατήρηση του συνολικού δείγματος. Δηλαδή $ R = t_{max}-t_{min} $ (ισχύει μόνο για μεταβλητές ποσοτικές)

\textbf{Διάμεσος $\delta$} του δείγματος ονομάζεται η μεσαία παρατήρηση εφόσον οι παρατηρήσεις έχουν διαταχθεί σε αύξουσα ή φθίνουσα σειρά, και ειναι το πλήθος τους περιττός αριθμός. Εαν το πλήθος είναι άρτιος αριθμός, τότε διάμεσος είναι ο μέσος όρος των δυο μεσαίων παρατηρήσεων εφόσον έχουν διαταχθεί οι παρατηρήσεις. (ισχύει μόνο για μεταβλητές ποσοτικές)

\textbf{Επικρατούσα τιμή} είναι η τιμή της μεταβλητής που εμφανίζεται περισσότερες φορές απο κάθε άλλη τιμή. Μπορούν να υπάρχουν περισσότερες απο μια επικρατούσες τιμές, αλλά πρέπει ναναι σίγουρα λιγότερες απο το δείγμα.

ΔΙΑΓΡΑΜΜΑΤΑ

\textbf{Ραβδόγραμμα} είναι ένα διάγραμμα όπου χρησιμοποιούνται συνήθως για ποιοτικές μεταβλητές, με στήλες που βρίσκονται σε οριζόντιο (κάθετο ραβδόγραμμα) ή κάθετο άξονα (οριζόντιο ραβδόγραμμα). Μπορεί το ραβδόγραμμα να είναι ραβδόγραμμα συχνοτήτων, σε αυτή την περίπτωση, το μέγεθος (μήκος ή ύψος) της κάθε ράβδου είναι η αντίστοιχη συχνότητα. Η' μπορεί ναναι ραβδόγραμμα σχετικών συχνοτήτων (ή σχετικών συχνοτήτων επι τοις εκατό), οπότε σε αυτή την περίπτωση το μέγεθος (μήκος ή ύψος) είναι αντίστοιχα η σχετική συχνότητα (ή σχετική συχνότητα επι τοις εκατό).

To \textbf{κυκλικό διάγραμμα} χρησιμοποιείται για την γραφική παράσταση ποιοτικών και ποσοτικών δεδομένων. Χρησιμοποιείται όταν οι διαφορετικές τιμές είναι σχετικά λίγες. Στο κυκλικό διάγραμμα επειδή είναι δύσκολο να αντιληφθούμε πολλές φορές ποιος τομέας ειναι μεγαλύτερος, δίδονται και αλλες πληροφορίες για κάθε τομέα, είτε η συχνότητα, είτε η σχετική συχνότητα είτε η σχετική συχνότητα επι τοις εκατό. Για τον υπολογισμό του κάθε τομέα, βρίσκουμε πόσες μοίρες πρέπει να είναι στον κύκλο, όπου το σύνολο είναι $ 360^0 $. Κάθε τομέας $ a_i $ έχει μοίρες $ a_i = f_i \cdot 360^0 $



\end{multicols}
\newpage
\section{Μεθοδολογία}
Πως κατασκευάζουμε ένα πίνακα συχνοτήτων, οι διάφορες στήλες, και πως υπολογίζονται
\begin{enumerate}
\item \textbf{Μεταβλητή $ X $} Βρίσκουμε ποιά ειναι η μεταβλητή $ X $
\item \textbf{Δυνατές τιμές της μεταβλητής $ X $} Εντοπίζουμε τις διαφορετικές τιμές που παίρνει η μεταβλητή $ X $. Για κάθε διαφορετική λοιπόν τιμή της $ X $ κάνουμε απο μια γραμμή στον πίνακα συχνοτήτων. Γράφουμε τις τιμές στην στήλη $ X_i $ κατα αύξουσα σειρά (εαν ειναι ποσοτική μεταβλητή) ή με όποια σειρά θέλουμε (εαν ειναι ποιοτική μεταβλητή)
\item \textbf{Συχνότητα $ \nu_i$} Σε κάθε γραμμή, μετράμε πόσες φορές εμφανίζεται η αντίστοιχη μεταβλητή, και αυτό το γράφουμε στην ίδια γραμμή στην στήλη $\nu_i$. Το άθροισμα όλων των συχνοτήτων στο τέλος ισούτε με τον πληθυσμό.
\item \textbf{Αθροιστική Συχνότητα $N_i$} Στην στήλη αυτή έχουμε το άθροισμα των συχνοτήτων $ \nu_i $ μέχρι την συγκεκριμένη γραμμή. Ενας άλλος τρόπος υπολογισμού ειναι ότι για κάθε $i, N_i=N_{i-1}+\nu_i$. Με απλά λόγια δηλαδή, ισουται με το επάνω $N_i$ συν το $\nu_i$ στην ίδια γραμμή.
\item \textbf{Σχετική συχνότητα $f_i$ } Για τον υπολογισμό της σχετικής συχνότητας χρειαζόμαστε το αντίστοιχο $\nu_i$ και τον πληθυσμό $\nu$. Ο υπολογισμός ειναι $ f_i=\dfrac{\nu_i}{\nu} $. Η σχέση αυτή χρησιμοποιείτε επίσης οποτεδήποτε γνωρίζουμε δυο απο τα τρία στοιχεία, και θέλουμε το τρίτο.
\item \textbf{Αθροιστική Σχετική συχνότητα $F_i$} Λειτουργεί με την ίδια λογική με την αθροιστική συχνότητα, αλλά χρησιμοποιώντας την στήλη $f_i$ και όχι την $\nu_i$.
\item \textbf{Σχετική συχνότητα επι τοις εκατό $ f_i\% $ }
\end{enumerate}

Πως βρίσκουμε την \textbf{διάμεσο} (ΜΟΝΟ ΓΙΑ ΠΟΣΟΤΙΚΕΣ ΜΕΤΑΒΛΗΤΕΣ)
\begin{enumerate}
	\item Εαν έχουμε παρατηρήσεις
	\begin{enumerate}
		\item Βάζουμε στην σειρά τις παρατηρήσεις, απο την μικρότερη τιμή προς την μεγαλύτερη
		\item Μετράμε τον πληθυσμό των παρατηρήσεων, έστω ότι είναι $ \nu $
		\item Εαν ο πληθυσμός ειναι άρτιος αριθμός (ζυγός), τότε παίρνουμε την $ \dfrac{\nu}{2} $ παρατήρηση, δηλαδή την $ t_{\frac{\nu}{2}} $ και την επόμενη της, δηλαδή την $ t_{\frac{\nu}{2}+1} $ και τις προσθέτουμε και το αποτέλεσμα το διαιρούμε δια δύο. Αυτό ειναι η διάμεσος $\delta$. Δηλαδή $ \delta=\dfrac{t_{\frac{\nu}{2}}+t_{\frac{\nu}{2}+1}}{2} $
		\item Εαν ομως ο πληθυσμός είναι περιττός αριθμός (μονός), τότε παίρνουμε ακριβώς την παρατήρηση που ειναι στην μέση, και αυτή ειναι η ακέραια τιμή του $ \dfrac{\nu}{2} $ συν 1. Δηλαδή $ \delta = t_{\lfloor \frac{\nu}{2} \rfloor +1} $
	\end{enumerate}
	\item Eαν έχουμε πίνακα συχνοτήτων, η λογική είναι ανάλογη. Υποθέτουμε ότι είναι σε αύξουσα διάταξη ως προς την τιμή της μεταβλητής $ x_i $. Βοηθάει επίσης να έχουμε την στήλη της αθροιστικής συχνότητας $ N_i $. 
	\begin{itemize}
		\item Βρίσκουμε τον πληθυσμό. Εαν ο πληθυσμός ειναι άρτιος αριθμός (ζυγός) τότε:
		\begin{enumerate}
			\item Βρίσκουμε την μεσαία τιμή, δηλαδή το $\dfrac{\nu}{2} $
			\item Απο την στήλη των αθροιστικών συχνοτήτων, στην οποία κάθε αριθμός δείχνει ποιά ειναι η τελευταία παρατήρηση κατα αύξοντα αριθμό, βρίσκουμε την γραμμή στην οποία πρέπει να ανήκει το $\dfrac{\nu}{2} $, δηλαδή ναναι ισο ή μικρότερο απο το $ N_i$ της γραμμής αλλά μεγαλύτερο απο αυτό της προηγούμενης. Την γραμμή στην οποία το βρηκαμε αυτό, παίρνουμε την τιμή $ x_i $ και αυτό είναι η μια παρατήρηση. Το ίδιο κάνουμε και για την αμέσως επόμενη παρατήρηση, βρίσκουμε και το δικό της $x_i$ που μπορεί ναναι στην ίδια γραμμή, ή στην επόμενη. Απο τις δυο αυτές παρατηρήσεις παίρνουμε τον μέσο όρο, αθροίζοντας τες και διαιρώντας με 2. Αυτό ειναι η διάμεσος.
		\end{enumerate}
		\item Εαν ο πληθυσμός ειναι περιττός αριθμός (μονός), τότε:
		\begin{enumerate}
			\item Βρίσκουμε ποιά ειναι η μεσαία παρατήρηση, υπολογίζοντας $\lfloor \frac{\nu}{2} \rfloor +1$
			\item Εκτελούμε την ίδια διαδικασία, όπως πιο πάνω, για να βρούμε σε ποιά γραμμή αντιστοιχεί η συγκεκριμένη παρατήρηση μέσω της στήλης $N_i$, και όταν βρούμε ποιά γραμμή ειναι, παίρνουμε την τιμή της $x_i$. Αυτή ειναι η διάμεσος
		\end{enumerate}
	\end{itemize}
\end{enumerate}



\newpage
\section{Προβλήματα}
\begin{enumerate}
\item Ρωτήσαμε 10 μέλη ενός γυμναστηρίου πόσες ωρες γυμνάζονται καθημερινά, και οι απαντήσεις που λάβαμε ήταν οι εξής: 2,1,3,3,2,1,3,4,5,1
\begin{enumerate}
\item Ποιές ειναι οι παρατηρήσεις;
\item Ποιά ειναι η μεταβλητή που εξετάζουμε;
\item Να κατασκευαστεί ο πίνακας συχνοτήτων με όλες τις στήλες
\end{enumerate}
\item Να συμπληρώσετε τον παρακάτω πίνακα
\begin{table}[H]
\begin{tabular}{|c|c|c|c|c|c|c|} \hline
$x_i$	& $ \nu_i $	& $ f_i $	& $ f_i\% $	& $ N_i $	& $ F_i $	& $ F_i\% $	\\ \hline
-1		& 3			&			&			&			&			&			\\ \hline
0		& 8			&			&			&			&			&			\\ \hline
2		& 5			&			&			&			&			&			\\ \hline
4		& 4			&			&			&			&			&			\\ \hline
ΣΥΝΟΛΟ	&			&			&			&			&			&			\\ \hline
\end{tabular}
\end{table}
\item Να συμπληρώσετε τον παρακάτω πίνακα
\begin{table}[H]
\begin{tabular}{|C{1.7cm}|C{1.7cm}|C{1.7cm}|C{1.7cm}|C{1.7cm}|C{1.7cm}|C{1.7cm}|} \hline

$x_i$	& $ \nu_i $	& $ f_i $	& $ f_i\% $	& $ N_i $	& $ F_i $	& $ F_i\% $	\\ \hline

3		& 			&			&			&			& $ 0,2 $	&			\\ \hline
7		& 15			& $ 0.3 $	& 30			&			&			&			\\ \hline
9		&			&			&			& 45			&			&			\\ \hline
10		&			&			&			&			&			&			\\ \hline
\textbf{ΣΥΝΟΛΟ}	&	&			&			&			&			&			\\ \hline
\end{tabular}
\end{table}

\item Εξετάστηκαν 30 μαθητές ως προς την ομάδα αίματος και είχαμε τα παρακάτω αποτελέσματα: 0,0,ΑΒ,Α,Β,Α,Β,0,Α,0,ΑΒ,0,Β,0,Α,Β,Α,0,Α,Β,Α,0,ΑΒ,0,Α,0,Β,0,Α,0
\begin{enumerate}
\item Να κατασκευαστεί ο πίνακας συχνοτήτων καθώς και οι στήλες με τις σχετικές συχνότητες
\item Να κατασκευαστεί ραβδόγραμμα συχνοτήτων
\item Να κατασκευαστεί κυκλικό διάγραμμα και να βρεθεί πόσες μοίρες αντιστοιχούν σε κάθε τομέα
\end{enumerate}

\item Δίνεται  ο παρακάτω πίνακας
\begin{table}[H]
\begin{tabular}{|C{1.7cm}|C{1.7cm}|C{1.7cm}|C{1.7cm}|C{1.7cm}|} \hline
$x_i$	& $ \nu_i $	& $ f_i\% $	& $ F_i\% $	& $ N_i $	\\ \hline
15 & 32 & 40 & & \\ \hline
17 & 22 & 27,5 & & \\ \hline
18 & 16 & 20 & & \\ \hline
20 & 10 & 12,5 & & \\ \hline
\textbf{ΣΥΝΟΛΟ} & & & & \\ \hline
\end{tabular}
\end{table}
\begin{enumerate}
\item Να συμπληρωθεί ο πίνακας
\item Να κατασκευαστεί το διάγραμμα συχνοτήτων και το αντίστοιχο πολύγωνο
\item Να κατασκευάσετε το πολύγωνο των αθροιστικών σχετικών συχνοτήτων επι τοις \%.
\item Ποιό ποσοστό των παρατηρήσεων είναι το πολύ 17;
\end{enumerate}

\item Δίδεται ο παρακάτω πίνακας
\begin{table}[H]
\begin{tabular}{|C{2cm}|C{2cm}|C{2cm}|C{2cm}|} \hline
$x_i$	& $ \nu_i $	& $ f_i $	& $ f_i\% $	\\ \hline
$x_1$   &            &           & 10        \\ \hline 
$x_2$   &            &           &           \\ \hline 
$x_3$   & 8          & 0,2       &           \\ \hline 
$x_4$   &            & 0,35      &           \\ \hline 
\textbf{ΣΥΝΟΛΟ}  &            & 1         & 100       \\ \hline
\end{tabular}
\end{table}

\item Σε ένα κυκλικό διάγραμμα παριστανεται ο αριθμός των θηραμάτων που πέτυχε ένας κυνηγός στην διάρκεια της κυνηγετικής περιόδου.
\begin{itemize}
\item Το 20\% των θηραμάτων ήταν τρυγόνια
\item 75 θηράματα ήταν τσίχλες
\item Η γωνία του κυκλικού τομέα που αντιστοιχεί στις πάπιες, είναι ιση με αυτή που αντιστοιχεί στα τρυγόνια
\item Το εμβαδόν του κυκλικού τομέα που αντιστοιχεί στους λαγούς είναι τετραπλάσιο απο αυτό που αντιστοιχεί στα αγριογούρουνα
\item Η γωνία που αντιστοιχεί στις τσίχλες ειναι ίση με $ 180^0 $
\end{itemize}
\begin{enumerate}
\item Να βρείτε πόσα θηράματα πέτυχε ο κυνηγός
\item Να βρείτε πόσα τρυγόνια και πόσες πάπιες πέτυχε ο κυνηγός
\item Να κατασκευάσετε το κυκλικό διάγραμμα σχετικών συχνοτήτων επι τοις εκατό
\end{enumerate}

\item Σε ένα κυκλικο διάγραμμα συχνοτήτων παρουσιάζονται οι βαθμοί που είχαν οι συμμετέχοντες (4,5,6,7)
\begin{itemize}
\item Η γωνία του κυκλικού τομέα που αντιστοιχεί στον βαθμό 4, ισούται με $ 36^0 $
\item Το $ \dfrac{1}{4} $ των υποψήφιων έγραψε 5
\item 80 υποψήφιοι είχαν βαθμό 6
\item Οι υποψήφιοι που έγραψαν 7 είναι τετραπλάσιοι απο αυτούς που έγραψαν 8
\end{itemize}
\begin{enumerate}
\item Να κατασκευάσετε πίνακα συχνοτήτων με τους βαθμούς που έγραψαν
\item Eαν για να πετύχεις στις εξετάσεις πρέπει να είσαι στο άνω 25\%, ποιός ειναι ο ελάχιστος βαθμός που πρέπει να γράψεις;
\end{enumerate}
\end{enumerate}

\end{document}